\documentclass[oneside, english, reqno]{amsart}
\usepackage[top=1.15in, bottom=1.15in, left=1.6in, right=1.6in]{geometry}
\usepackage[T1]{fontenc}
\usepackage[latin1]{inputenc}
\pagestyle{plain}
\usepackage{amsmath, amsthm, amssymb, amsfonts}
\usepackage{multirow, colortbl, graphicx, hyperref, wrapfig}
\usepackage{tikz, empheq}
\usepackage{lscape}
\renewcommand{\qedsymbol}{$\blacksquare$}

% Page numbering
\usepackage{fancyhdr}
\fancypagestyle{plain}{%
\fancyhf{} % clear all header and footer fields
\fancyfoot[C]{\sffamily\fontsize{10pt}{50pt}\selectfont\thepage} % except the center
\renewcommand{\headrulewidth}{0pt}
\renewcommand{\footrulewidth}{0pt}}
\pagestyle{plain}

% Mathematical headings
\newtheoremstyle{plain}{20pt}{15pt}{\slshape}{0pt}{\bfseries}{}{4mm}{}
\newtheoremstyle{definition}{20pt}{20pt}{\slshape}{0pt}{\bfseries}{}{4mm}{}
\theoremstyle{plain}
\newtheorem{thm}{Theorem}[section]
\theoremstyle{plain}
\newtheorem{lemma}[thm]{Lemma}
\theoremstyle{plain}
\newtheorem{theorem}[thm]{Theorem}
\theoremstyle{plain}
\newtheorem{corollary}[thm]{Corollary}
\theoremstyle{definition}
\newtheorem{definition}[thm]{Definition}
\theoremstyle{definition}
\newtheorem{example}[thm]{Example}
\theoremstyle{definition}
\newtheorem{conjecture}[thm]{Conjecture}

\usepackage{babel}

% Create a table
\usepackage{tabularx}
\usepackage{booktabs}
\newcolumntype{L}[1]{>{\raggedright\let\newline\\\arraybackslash\hspace{0pt}}p{#1}}
\newcolumntype{C}[1]{>{\centering\let\newline\\\arraybackslash\hspace{0pt}}p{#1}}
\newcolumntype{R}[1]{>{\raggedleft\let\newline\\\arraybackslash\hspace{0pt}}p{#1}}

% Settings for fbox
\setlength\fboxsep{5pt}
\setlength\fboxrule{1pt}

% Add vertical spaces after paragraphs
\setlength{\parskip}{2pt}%
\setlength{\parindent}{15pt}%

\makeatletter
% Add vertical spaces around equations
\g@addto@macro\normalsize{%
  \setlength\abovedisplayskip{10pt}
  \setlength\belowdisplayskip{10pt}
  \setlength\abovedisplayshortskip{10pt}
  \setlength\belowdisplayshortskip{10pt}
}
% Add a new line after subsection title
\def\subsection{\@startsection{subsection}{3}%
  \z@{2\linespacing}{.3\linespacing}%
  {\normalfont}}
% Add a new line after subsubsection title
\def\subsubsection{\@startsection{subsubsection}{3}%
  \z@{2\linespacing}{.3\linespacing}%
  {\normalfont}}
\makeatother

% Link an equation to a particular section
\numberwithin{equation}{section}

% Link a figure to a particular section
\usepackage{chngcntr}
\counterwithin{figure}{section}

% Create an equation number for align* if desired
\newcommand\numberthis{\addtocounter{equation}{1}\tag{\theequation}}

% Circled numbers
\usepackage{tikz}
\newcommand*\circled[1]{\tikz[baseline=(char.base)]{
            \node[shape=circle,draw,inner sep=1.25pt] (char) {#1};}}

\allowdisplaybreaks[1]

\begin{document}

\section{How to set up a truss}
\label{sec:1}

\subsection{Define the nodes}
\label{sec:1.1}
You can define the nodes that make up the structure by clicking the \underline{Add}, \underline{Edit}, or \underline{Remove} button under the \underline{Nodes} section. The nodes are called ``keypoints'' in Ansys.

Please note that the GUI will automatically assign the node indices. The node indices are always consecutive integers, starting with the number 1. \vspace{20pt}

To add a node, please specify the following:
\begin{equation*}
		\begin{array}{ll}
				\mbox{x-coordinate} & \mbox{the (global) x-coordinate of the node} \\[6pt]
				\mbox{y-coordinate} & \mbox{the (global) y-coordinate of the node}
		\end{array}
\end{equation*} \vspace{10pt}

To edit an existing node, please specify the following:
\begin{equation*}
		\begin{array}{ll}
				\mbox{Index of the node} & \mbox{the index of the node that you want to modify} \\[8pt]
				\mbox{x-coordinate} & \mbox{the (global) x-coordinate of the node} \\[8pt]
				\mbox{y-coordinate} & \mbox{the (global) y-coordinate of the node}
		\end{array}
\end{equation*} \vspace{10pt}

To remove an existing node, please specify the following:
\begin{equation*}
		\begin{array}{ll}
				\mbox{Index of the node} & \mbox{the index of the node that you want to remove}
		\end{array}
\end{equation*} \vspace{1pt}

\subsection{Define the elements}
\label{sec:1.2}
You can define the elements that make up the structure by clicking the \underline{Add} or \underline{Remove} button under the \underline{Elements} section. The truss elements are called ``lines'' in Ansys.

Please note that the GUI will automatically assign the element indices. The element indices are always consecutive integers, starting with the number 1. \vspace{20pt}

To add an element, please specify the following:
\begin{equation*}
		\begin{array}{ll}
				\mbox{Indices of the two nodes} & \mbox{the indices of the start node and the end node} \\[6pt]
				\mbox{Young's modulus} & \mbox{a material parameter} \\[6pt]
				\mbox{Cross-sectional area} & \mbox{a material parameter}
		\end{array}
\end{equation*} \vspace{10pt}

To remove an existing element, please specify the following:
\begin{equation*}
		\begin{array}{ll}
				\mbox{Index of the element} & \mbox{the index of the element that you want to remove}
		\end{array}
\end{equation*}


\newpage
\subsection{Define the boundary conditions (BCs)}
\label{sec:1.3}
You can define the boundary conditions on the nodes by clicking the \underline{Add} or \underline{Remove} button under the \underline{Boundary conditions} section.

Please note that you must specify the BC components in \(x\)- and \(y\)-directions together when adding a BC. When you export the assembly file, all nodes without a boundary condition will be assigned a zero force BC in the \(x\)- and \(y\)-directions. \vspace{20pt}

To add a boundary condition, please specify the following:
\begin{equation*}
		\begin{array}{ll}
				\mbox{Index of the node} & \mbox{the index of the node where you want to impose the BC} \\[6pt]
				\mbox{Type of boundary condition} & \mbox{indicate whether you want a displacement or force BC} \\[6pt]
				\mbox{Value of the x-component} & \mbox{the amount of displacement/force in the x-direction} \\[6pt]
				\mbox{Value of the y-component} & \mbox{the amount of displacement/force in the y-direction} \\[6pt]
		\end{array}
\end{equation*} \vspace{1pt}

Some common boundary conditions: \vspace{6pt}

To specify a fixed pin, you would select ``Displacement, Displacement'' for the type of boundary condition, and enter the number 0 for the value of the x-component and the value of the y-component. (This is an example of a pure Dirichlet BC.) \vspace{6pt}

To specify a node with no external force, you would select ``Force, Force'' for the type of boundary condition, and enter the number 0 for the value of the x-component and the value of the y-component. Recall that the GUI can assign a zero force BC for you if you leave the node without a BC specified. (This is an example of a pure Neumann BC.) \vspace{6pt}

To specify a roller pin that is free to move vertically, select ``Displacement, Force'' for the type of boundary condition, and enter the number 0 for the value of the x-component and the value of the y-component. (This is an example of a mixed BC.) \vspace{6pt}

To specify a roller pin that is free to move horizontally, select ``Force, Displacement'' for the type of boundary condition, and enter 0 for the value of the x-component and the value of the y-component. (This is also an example of a mixed BC.) \vspace{6pt}

(Currently, the GUI does not support a roller pin on an inclined surface.) \vspace{25pt}

To remove a boundary condition, please specify the following:
\begin{equation*}
		\begin{array}{ll}
				\mbox{Index of the node} & \mbox{the index of the node where you want to remove the BC}
		\end{array}
\end{equation*} \vspace{1pt}



\subsection{Display options}
\label{sec:1.4}
You can indicate whether you want to see the node indices, the element indices, the arrows representing the BCs, and the grid on the display.

To do this, you can press Ctrl + 1 (2, 3, or 4), or go to the \underline{Tools} menu and click on \underline{Display options}, followed by \underline{Node index}, etc. \vspace{1pt}


\newpage
\subsection{Save your workspace, come back later}
\label{sec:1.5}
You can save your current work as a file (it is saved as a .mat file), and come back later to continue working on your structure.

To do this, you can press Ctrl + s, or go to the \underline{File} menu and click on \underline{Save workspace}. Please enter the name of the file, and select the directory under which you want to save the file. Click the \underline{Save} button.

To open a workspace file, you can press Ctrl + o, or go to \underline{File} menu and click on \underline{Open workspace}. Please check that you select a workspace file, and not an assembly file (which have ``\_assembly'' appended to the name of the file) or any other .mat file. The GUI does not check whether you have entered a valid .mat file. \vspace{1pt}


\subsection{Create the assembly file}
\label{sec:1.6}
Once you have specified the nodes, elements, and BCs that make up the structure, you can organize all the information needed for assembly by creating an assembly file. To do this, please go to the \underline{File} menu and click on \underline{Export}, followed by \underline{Assembly file}. The GUI will then automatically create the assembly file in the same directory as your workspace file, and name it by appending ``\_assembly'' to the workspace file name.

The assembly file is saved under two formats: one as a single .mat file, and the other as four .txt files. While we recommend using the .mat file with Matlab (you load the file with one statement, and the values are automatically read and stored in memory), you may use instead the .txt files. You will have to use the .txt files if you are using another language for assembly and postprocessing.

The .mat file contains 4 variables, named nodes, elements, BCs\_displacement, and BCs\_force. Each one is a double array (while technically not correct, you may think of it as a matrix), where the rows correspond to a node, an element, a displacement BC, or a force BC, and the columns correspond to certain information.

We list what the columns stand for here:

\begin{equation*}
		\begin{array}{ll}
				\mbox{nodes} & \mbox{[x-coordinate, y-coordinate]} \\[6pt]
				\mbox{elements} & \mbox{[start node index, end node index, Young's modulus, cross-sectional area]} \\[6pt]
				\mbox{BCs\_displacement} & \mbox{[node index, coordinate index, BC value]} \\[6pt]
				\mbox{BCs\_force} & \mbox{[node index, coordinate index, BC value]}
		\end{array}
\end{equation*} \vspace{6pt}

Note that we do not store the node indices in the nodes array, the element indices in the elements array, etc., because we assume that the indices are consecutive integers starting with 1. To find the number of nodes in the nodes array (and so on), we can use Matlab's size routine, which will return the dimensions of a double array. Please see the next page for the correct syntax.

The .txt files always begin with the number of lines (i.e. the number of nodes, etc.) that will follow afterwards, but contain otherwise the same information in the same order as shown above. The multiple values in a line are separated by a whitespace, i.e. a whitespace is the delimiter.


\newpage
\section{Some useful Matlab routines}
\label{sec:2}
When you write your assembly routine in Matlab, you may find these built-in routines useful. If you have an idea of what needs to be done but aren't sure how to write it in Matlab, reading Matlab's online documentation or looking for examples on Google are a great help. \vspace{15pt}

\noindent load(`str'), where str is a string (must be enclosed by single quotation marks as shown) \\[6pt]
\mbox{\hspace{0.5cm}} You must use this routine to load the .mat assembly file; the four variables are then read automatically, and can be used anywhere in your code following the load statement. str is the full path to the .mat assembly file, i.e. it is the combination (concatenation) of the directory path and the name of the assembly file. If you have the .mat assembly file in the same directory as your assembly routine, then entering just the name of the assembly file in place of str is enough.

Near line 552 in main.m is an example of the load routine (it loads the .mat workspace file); the additional arguments indicate which variables we want to load from the file. You may leave out these arguments if you want to read all variables in the file. \vspace{15pt}

\noindent A(a, b), where A is a matrix, and a and b are column vectors (of positive integers) \\[6pt]
\mbox{\hspace{0.5cm}} This returns a submatrix of A, with the rows specified by the index vector a, and the columns specified by the index vector b. To get a submatrix with all the rows but some of the columns of A, we use a colon and type in A(:, b). To get a submatrix with all the columns but some of the rows of A, we would type in A(a, :). \vspace{15pt}

\noindent find(expr), where expr is a logical expression involving matrices vectors, and/or scalars \\[6pt]
\mbox{\hspace{0.5cm}} This returns a vector of indices where the logical expression is a true statement. \vspace{15pt}

\noindent rank(A), where A is a matrix \\[6pt]
\mbox{\hspace{0.5cm}} This returns the rank of A. Recall that, if \(A\) is a \(n \times n\) square matrix appearing in a matrix equation \(A\vec{x} = \vec{b}\) (assume that \(\vec{b}\) is in the column space of \(A\), i.e. a solution exists), then \(n\) minus the rank tells us the minimum number of constraints that we need to impose on the vector \(\vec{x}\) if we want \(\vec{x}\) to be a unique solution. \vspace{15pt}

\noindent size(A, 1), where A is a matrix, vector, or scalar \\[6pt]
\mbox{\hspace{0.5cm}} By setting the second argument to be 1, the routine finds the row dimension of A. \vspace{15pt}

\noindent sort(a), where a is a column vector \\[6pt]
\mbox{\hspace{0.5cm}} This returns a column vector of the same size as a, but with the entries sorted in ascending order. You can also use [b, permutation] = sort(a), which returns two vectors: b is the sorted vector, and permutation is a vector that indicates how the rows of a have been permuted (switched around) in order to get to b. \vspace{10pt}


\end{document}